% !TEX TS-program = pdflatex
% !TEX encoding = UTF-8 Unicode

\documentclass[11pt]{article} % use larger type; default would be 10pt

\usepackage[utf8]{inputenc} % set input encoding (not needed with XeLaTeX)

%%% PAGE DIMENSIONS
\usepackage{geometry}
\geometry{a4paper}

\usepackage{graphicx} % support the \includegraphics command and options

% \usepackage[parfill]{parskip} % Activate to begin paragraphs with an empty line rather than an indent

%%% PACKAGES
\usepackage{booktabs} % for much better looking tables
\usepackage{array} 	  % for better arrays (eg matrices) in maths
\usepackage{paralist} % very flexible & customisable lists (eg. enumerate/itemize, etc.)
\usepackage{verbatim} % adds environment for commenting out blocks of text & for better verbatim
\usepackage{subfig}   % make it possible to include more than one captioned figure/table in a single float
\usepackage{listings} % for code presentation
% These packages are all incorporated in the memoir class to one degree or another...

%%% HEADERS & FOOTERS
\usepackage{fancyhdr} % This should be set AFTER setting up the page geometry
\pagestyle{fancy} % options: empty , plain , fancy
\renewcommand{\headrulewidth}{0pt} % customise the layout...
\lhead{}\chead{}\rhead{} % Setup Header
\lfoot{}\cfoot{\thepage}\rfoot{} % Setup Footer

%%% SECTION TITLE APPEARANCE
\usepackage{sectsty}
\allsectionsfont{\sffamily\mdseries\upshape} % (See the fntguide.pdf for font help)
% (This matches ConTeXt defaults)

%%% ToC (table of contents) APPEARANCE
%\usepackage[nottoc,notlof,notlot]{tocbibind} % Put the bibliography in the ToC
%\usepackage[titles,subfigure]{tocloft} % Alter the style of the Table of Contents
%\renewcommand{\cftsecfont}{\rmfamily\mdseries\upshape}
%\renewcommand{\cftsecpagefont}{\rmfamily\mdseries\upshape} % No bold!

%%% END Article customizations

%%% The "real" document content comes below...

\title{Report 0}
\author{Arthur BRUGIERE}
%\date{} % Activate to display a given date or no date (if empty),
         % otherwise the current date is printed 

\begin{document}
\maketitle

\section{How do you compile the labwork project?}

Any {\it cmake} project is typically built as follows:

\lstset{language=Bash}
\begin{lstlisting}
arthurbrugiere@ictserver3:~/advancedhpc2018/labwork$ mkdir build
arthurbrugiere@ictserver3:~/advancedhpc2018/labwork$ cd build
arthurbrugiere@ictserver3:~/advancedhpc2018/labwork/build$ cmake ..
arthurbrugiere@ictserver3:~/advancedhpc2018/labwork/build$ make -j
\end{lstlisting}

\section{What do you need to compile the project?}

To build it, you will need the following dependencies:

\begin{itemize}
	\item OS: GNU/Linux or macOS
	\item CUDA SDK 7+ 
	\item Compiler : 
			\subitem  {\it nvcc} (bundled with CUDA SDK)
			\subitem {\it gcc} 4.8+ (for C++11 standard support).
	\item {\it libjpeg} to encode/decode JPEG files
\end{itemize}

Optional:

\begin{itemize}
\item  {\it cmake} can help you build even easier. A {\it CMakeLists.txt} is provided. If not, manual {\it nvcc} is still applicable.
\end{itemize}

\section{Write command and output for your compilation}

\lstset{language=Bash}
\begin{lstlisting}
arthurbrugiere@ictserver3:~/advancedhpc2018/labwork/build$ ./labwork 1 ../data/eiffel.jpg
USTH ICT Master 2018, Advanced Programming for HPC.
Warming up...
Starting labwork 1
labwork 1 CPU ellapsed 3230.3ms
labwork 1 GPU ellapsed 793.8ms
labwork 1 ellapsed 793.8ms
\end{lstlisting}

\end{document}
