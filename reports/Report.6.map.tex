% !TEX TS-program = pdflatex
% !TEX encoding = UTF-8 Unicode

\documentclass[11pt]{article} % use larger type; default would be 10pt

\usepackage[utf8]{inputenc} % set input encoding (not needed with XeLaTeX)

%%% PAGE DIMENSIONS
\usepackage{geometry}
\geometry{a4paper}

\usepackage{graphicx} % support the \includegraphics command and options

% \usepackage[parfill]{parskip} % Activate to begin paragraphs with an empty line rather than an indent

%%% PACKAGES
\usepackage{booktabs} % for much better looking tables
\usepackage{array} 	  % for better arrays (eg matrices) in maths
\usepackage{paralist} % very flexible & customisable lists (eg. enumerate/itemize, etc.)
\usepackage{verbatim} % adds environment for commenting out blocks of text & for better verbatim
\usepackage{subfig}   % make it possible to include more than one captioned figure/table in a single float

% These packages are all incorporated in the memoir class to one degree or another...

%%% HEADERS & FOOTERS
\usepackage{fancyhdr} % This should be set AFTER setting up the page geometry
\pagestyle{fancy} % options: empty , plain , fancy
\renewcommand{\headrulewidth}{0pt} % customise the layout...
\lhead{}\chead{}\rhead{} % Setup Header
\lfoot{}\cfoot{\thepage}\rfoot{} % Setup Footer

%%% SECTION TITLE APPEARANCE
\usepackage{sectsty}
\allsectionsfont{\sffamily\mdseries\upshape} % (See the fntguide.pdf for font help)
% (This matches ConTeXt defaults)

%%% ToC (table of contents) APPEARANCE
%\usepackage[nottoc,notlof,notlot]{tocbibind} % Put the bibliography in the ToC
%\usepackage[titles,subfigure]{tocloft} % Alter the style of the Table of Contents
%\renewcommand{\cftsecfont}{\rmfamily\mdseries\upshape}
%\renewcommand{\cftsecpagefont}{\rmfamily\mdseries\upshape} % No bold!

%%% DEV & CODE 
\usepackage{xcolor}
\usepackage{listings} % for code presentation

\definecolor{mGreen}{rgb}{0,0.6,0}
\definecolor{mGray}{rgb}{0.5,0.5,0.5}
\definecolor{mPurple}{rgb}{0.58,0,0.82}
\definecolor{backgroundColour}{rgb}{0.95,0.95,0.92}

\lstdefinestyle{CStyle}{
    backgroundcolor=\color{backgroundColour},   
    commentstyle=\color{mGreen},
    keywordstyle=\color{magenta},
    numberstyle=\tiny\color{mGray},
    stringstyle=\color{mPurple},
    basicstyle=\footnotesize,
    breakatwhitespace=false,         
    breaklines=true,                 
    captionpos=b,                    
    keepspaces=true,                 
    numbers=left,                    
    numbersep=5pt,                  
    showspaces=false,                
    showstringspaces=false,
    showtabs=false,                  
    tabsize=2,
    language=C
}

%%% END Article customizations

%%% The "real" document content comes below...

\title{Report 6}
\author{Arthur BRUGIERE}
%\date{} % Activate to display a given date or no date (if empty),
         % otherwise the current date is printed 

\begin{document}
\maketitle

\section{Explain how you implement the labworks}

Because each processing that we had to implement in this labwork are Map Processing, each one was implemented in different Kernel. Each kernel is called with the most optimized parameters (2D blocks of 32x32 threads on a 2D grid). 

\subsection{Binarization}

The CPU call the kernel and all the processing work is on the GPU side :

\begin{lstlisting}[style=CStyle]
//Process pixel
output[tid].z = output[tid].y = output[tid].x = 
		(((int)(input[tid].x + input[tid].y + input[tid].z) / 3)/127)*255;
\end{lstlisting}

\subsection{Brightness control}

The CPU call the kernel and all the processing work is on the GPU side :

\begin{lstlisting}[style=CStyle]
//Process pixel
output[tid].x = min(255, max(0, input[tid].x + value));
output[tid].y = min(255, max(0, input[tid].y + value));
output[tid].z = min(255, max(0, input[tid].z + value));  
\end{lstlisting}

\subsection{Blending images}

For this kernel, it had a little difficulty : I had to get a second image. I've did it on the CPU side and send my second image in the same way than the first one.

\begin{lstlisting}[style=CStyle]
uchar3 *secondImg;

cudaMalloc(&secondImg, pixelCount * sizeof(uchar3));
cudaMemcpy(secondImg, inputSecondImage->buffer, pixelCount * sizeof(uchar3), cudaMemcpyHostToDevice);
\end{lstlisting}

After that, the CPU call the kernel and all the processing work is on the GPU side :

\begin{lstlisting}[style=CStyle]
//Process pixel
output[tid].x = (weight * (double)input[tid].x) 
			+ ((1.0 - weight) * (double)secondImg[tid].x);
output[tid].y = (weight * (double)input[tid].y) 
			+ ((1.0 - weight) * (double)secondImg[tid].y);
output[tid].z = (weight * (double)input[tid].z) 
			+ ((1.0 - weight) * (double)secondImg[tid].z);
\end{lstlisting}

%\section{Try experimenting with different 2D block size values}

\end{document}
