% !TEX TS-program = pdflatex
% !TEX encoding = UTF-8 Unicode

\documentclass[11pt]{article} % use larger type; default would be 10pt

\usepackage[utf8]{inputenc} % set input encoding (not needed with XeLaTeX)

%%% PAGE DIMENSIONS
\usepackage{geometry}
\geometry{a4paper}

\usepackage{graphicx} % support the \includegraphics command and options

% \usepackage[parfill]{parskip} % Activate to begin paragraphs with an empty line rather than an indent

%%% PACKAGES
\usepackage{booktabs} % for much better looking tables
\usepackage{array} 	  % for better arrays (eg matrices) in maths
\usepackage{paralist} % very flexible & customisable lists (eg. enumerate/itemize, etc.)
\usepackage{verbatim} % adds environment for commenting out blocks of text & for better verbatim
\usepackage{subfig}   % make it possible to include more than one captioned figure/table in a single float

% These packages are all incorporated in the memoir class to one degree or another...

%%% HEADERS & FOOTERS
\usepackage{fancyhdr} % This should be set AFTER setting up the page geometry
\pagestyle{fancy} % options: empty , plain , fancy
\renewcommand{\headrulewidth}{0pt} % customise the layout...
\lhead{}\chead{}\rhead{} % Setup Header
\lfoot{}\cfoot{\thepage}\rfoot{} % Setup Footer

%%% SECTION TITLE APPEARANCE
\usepackage{sectsty}
\allsectionsfont{\sffamily\mdseries\upshape} % (See the fntguide.pdf for font help)
% (This matches ConTeXt defaults)

%%% ToC (table of contents) APPEARANCE
%\usepackage[nottoc,notlof,notlot]{tocbibind} % Put the bibliography in the ToC
%\usepackage[titles,subfigure]{tocloft} % Alter the style of the Table of Contents
%\renewcommand{\cftsecfont}{\rmfamily\mdseries\upshape}
%\renewcommand{\cftsecpagefont}{\rmfamily\mdseries\upshape} % No bold!

%%% DEV & CODE 
\usepackage{xcolor}
\usepackage{listings} % for code presentation

\definecolor{mGreen}{rgb}{0,0.6,0}
\definecolor{mGray}{rgb}{0.5,0.5,0.5}
\definecolor{mPurple}{rgb}{0.58,0,0.82}
\definecolor{backgroundColour}{rgb}{0.95,0.95,0.92}

\lstdefinestyle{CStyle}{
    backgroundcolor=\color{backgroundColour},   
    commentstyle=\color{mGreen},
    keywordstyle=\color{magenta},
    numberstyle=\tiny\color{mGray},
    stringstyle=\color{mPurple},
    basicstyle=\footnotesize,
    breakatwhitespace=false,         
    breaklines=true,                 
    captionpos=b,                    
    keepspaces=true,                 
    numbers=left,                    
    numbersep=5pt,                  
    showspaces=false,                
    showstringspaces=false,
    showtabs=false,                  
    tabsize=2,
    language=C
}

%%% END Article customizations

%%% The "real" document content comes below...

\title{Report 9}
\author{Arthur BRUGIERE}
%\date{} % Activate to display a given date or no date (if empty),
         % otherwise the current date is printed 

\begin{document}
\maketitle

\section{Explain how you implement the labworks}

\subsection{Histogram}

In my first kernel I transformed my image in grayscale (of type {\it uchar3}) and save a more simple array of that grayscaled image (in type {\it int}).

\begin{lstlisting}[style=CStyle]
__global__ void grayscaleImgAndHisto(uchar3 *input, uchar3 *output, int *histo, int imgWidth, int imgHeight) {
    
    [...]

    // Outputs
    output[gtid].x = output[gtid].y = output[gtid].z = grayPx;
    histo[gtid] = grayPx;
}
\end{lstlisting}

With that simple histogram, I process my image line by line and save every little histogram in a structure of histograms. To achieve that, I use a strange structure with only one dimension of X threads ({\it X == number of row of the image}).

\begin{lstlisting}[style=CStyle]
typedef struct {
  unsigned int histogram[256];
} arrayOfHistograms;
__global__ void localGatherHisto(int *input, int imgWidth, arrayOfHistograms *arrayOfHisto) {
    // One thread(histo) / row
    unsigned int localHisto[256] = {0};
    
    //Calculate row histo
    for(int i = 0; i < imgWidth; i++)
        localHisto[ input[ blockIdx.y*imgWidth + i] ]++;
    
    //Store to SoA histo
    memcpy(arrayOfHisto[blockIdx.y].histogram, localHisto, sizeof(int)*256);    
}
void Labwork::labwork9_GPU() {

	[...]

    localGatherHisto<<<dim3(1, inputImage->height, 1), dim3(1,1,1)>>>(devHisto, inputImage->width, arrayOfHisto);

    [..]
}
\end{lstlisting}

Finally for that part, I reduce all of that littles histograms in a single histogram which is the one of my image.

\subsection{Histogram Equalization}

Finally, I calculate the CDF in serial way (so one loop over my histogram and no optimization) and apply this new value in a mapping kernel. Nothing special to show here.

\section{Explain and measure speedup, if you have performance optimizations}

As in the previous labwork, I optimize my algorithm on my memory usage (by the usage of my variables) only.

I also know that my reduction could be way more efficient, but I haven't get enough time to work on it.

\end{document}
