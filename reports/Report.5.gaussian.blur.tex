% !TEX TS-program = pdflatex
% !TEX encoding = UTF-8 Unicode

\documentclass[11pt]{article} % use larger type; default would be 10pt

\usepackage[utf8]{inputenc} % set input encoding (not needed with XeLaTeX)

%%% PAGE DIMENSIONS
\usepackage{geometry}
\geometry{a4paper}

\usepackage{graphicx} % support the \includegraphics command and options

% \usepackage[parfill]{parskip} % Activate to begin paragraphs with an empty line rather than an indent

%%% PACKAGES
\usepackage{booktabs} % for much better looking tables
\usepackage{array} 	  % for better arrays (eg matrices) in maths
\usepackage{paralist} % very flexible & customisable lists (eg. enumerate/itemize, etc.)
\usepackage{verbatim} % adds environment for commenting out blocks of text & for better verbatim
\usepackage{subfig}   % make it possible to include more than one captioned figure/table in a single float
\usepackage{listings} % for code presentation
% These packages are all incorporated in the memoir class to one degree or another...

%%% HEADERS & FOOTERS
\usepackage{fancyhdr} % This should be set AFTER setting up the page geometry
\pagestyle{fancy} % options: empty , plain , fancy
\renewcommand{\headrulewidth}{0pt} % customise the layout...
\lhead{}\chead{}\rhead{} % Setup Header
\lfoot{}\cfoot{\thepage}\rfoot{} % Setup Footer

%%% SECTION TITLE APPEARANCE
\usepackage{sectsty}
\allsectionsfont{\sffamily\mdseries\upshape} % (See the fntguide.pdf for font help)
% (This matches ConTeXt defaults)

%%% ToC (table of contents) APPEARANCE
%\usepackage[nottoc,notlof,notlot]{tocbibind} % Put the bibliography in the ToC
%\usepackage[titles,subfigure]{tocloft} % Alter the style of the Table of Contents
%\renewcommand{\cftsecfont}{\rmfamily\mdseries\upshape}
%\renewcommand{\cftsecpagefont}{\rmfamily\mdseries\upshape} % No bold!

%%% END Article customizations

%%% The "real" document content comes below...

\title{Report 5}
\author{Arthur BRUGIERE}
%\date{} % Activate to display a given date or no date (if empty),
         % otherwise the current date is printed 

\begin{document}
\maketitle

\section{Explain how you implement the Gaussian Blur filter}

The Gaussian blur is a type of image-blurring filter that uses a Gaussian function (which also expresses the normal distribution in statistics) for calculating the transformation to apply to each pixel in the image.

So, to implement it, I have create a kernel which applied for each pixels the convolution matrix below.

\begin{table}[h]
\begin{center}
  \begin{tabular}{ | c | c | c | c | c | c | c | }
    \hline
	0 & 0 & 1 & 2 & 1 & 0 & 0 \\ \hline
	0 & 3 & 13 & 22 & 13 & 3 & 0 \\ \hline
	1 & 13 & 59 & 97 & 59 & 13 & 1 \\ \hline
	2 & 22 & 97 & 159 & 97 & 22 & 2 \\ \hline
	1 & 13 & 59 & 97 & 59 & 13 & 1 \\ \hline
	0 & 3 & 13 & 22 & 13 & 3 & 0 \\ \hline
	0 & 0 & 1 & 2 & 1 & 0 & 0 \\
    \hline
  \end{tabular}
   \caption{Convolution matrix}
\end{center}
\end{table}

As on all the previous labworks, each pixel is treated by a unique thread.

\section{Try experimenting with different 2D block size values}

\section{Plot a graph of block size vs speedup (with/without shared memory)}

\end{document}
