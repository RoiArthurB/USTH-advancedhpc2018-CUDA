% !TEX TS-program = pdflatex
% !TEX encoding = UTF-8 Unicode

\documentclass[11pt]{article} % use larger type; default would be 10pt

\usepackage[utf8]{inputenc} % set input encoding (not needed with XeLaTeX)

%%% PAGE DIMENSIONS
\usepackage{geometry}
\geometry{a4paper}

\usepackage{graphicx} % support the \includegraphics command and options

% \usepackage[parfill]{parskip} % Activate to begin paragraphs with an empty line rather than an indent

%%% PACKAGES
\usepackage{booktabs} % for much better looking tables
\usepackage{array} 	  % for better arrays (eg matrices) in maths
\usepackage{paralist} % very flexible & customisable lists (eg. enumerate/itemize, etc.)
\usepackage{verbatim} % adds environment for commenting out blocks of text & for better verbatim
\usepackage{subfig}   % make it possible to include more than one captioned figure/table in a single float

% These packages are all incorporated in the memoir class to one degree or another...

%%% HEADERS & FOOTERS
\usepackage{fancyhdr} % This should be set AFTER setting up the page geometry
\pagestyle{fancy} % options: empty , plain , fancy
\renewcommand{\headrulewidth}{0pt} % customise the layout...
\lhead{}\chead{}\rhead{} % Setup Header
\lfoot{}\cfoot{\thepage}\rfoot{} % Setup Footer

%%% SECTION TITLE APPEARANCE
\usepackage{sectsty}
\allsectionsfont{\sffamily\mdseries\upshape} % (See the fntguide.pdf for font help)
% (This matches ConTeXt defaults)

%%% ToC (table of contents) APPEARANCE
%\usepackage[nottoc,notlof,notlot]{tocbibind} % Put the bibliography in the ToC
%\usepackage[titles,subfigure]{tocloft} % Alter the style of the Table of Contents
%\renewcommand{\cftsecfont}{\rmfamily\mdseries\upshape}
%\renewcommand{\cftsecpagefont}{\rmfamily\mdseries\upshape} % No bold!

%%% DEV & CODE 
\usepackage{xcolor}
\usepackage{listings} % for code presentation

\definecolor{mGreen}{rgb}{0,0.6,0}
\definecolor{mGray}{rgb}{0.5,0.5,0.5}
\definecolor{mPurple}{rgb}{0.58,0,0.82}
\definecolor{backgroundColour}{rgb}{0.95,0.95,0.92}

\lstdefinestyle{CStyle}{
    backgroundcolor=\color{backgroundColour},   
    commentstyle=\color{mGreen},
    keywordstyle=\color{magenta},
    numberstyle=\tiny\color{mGray},
    stringstyle=\color{mPurple},
    basicstyle=\footnotesize,
    breakatwhitespace=false,         
    breaklines=true,                 
    captionpos=b,                    
    keepspaces=true,                 
    numbers=left,                    
    numbersep=5pt,                  
    showspaces=false,                
    showstringspaces=false,
    showtabs=false,                  
    tabsize=2,
    language=C
}

%%% END Article customizations

%%% The "real" document content comes below...

\title{Report 10}
\author{Arthur BRUGIERE}
%\date{} % Activate to display a given date or no date (if empty),
         % otherwise the current date is printed 

\begin{document}
\maketitle

\section{Explain how you implement the labworks}

\subsection{First kernel : HSV}

To implement this labwork I have, firstly, get the V value from the HSV SoA of the image. To do it, I have implemented a simplified kernel of {\it RGB2HSV}.

\begin{lstlisting}[style=CStyle]
__global__ void RGB2V(uchar3 *in, double *out, int imgWidth, int imgHeight) {
    //Calculate tid
    unsigned int tidx = threadIdx.x + blockIdx.x * blockDim.x;
    unsigned int tidy = threadIdx.y + blockIdx.y * blockDim.y;
    if (tidx >= imgWidth || tidy >= imgHeight) return;
    
    int tid =  tidx + (tidy * imgWidth);
    
    out[tid] = max((double)in[tid].x / 255.0, max((double)in[tid].y / 255.0, (double)in[tid].z / 255.0));
}
\end{lstlisting}

\subsection{Second kernel : Process kuwahara filter}

After that, I have process the 4 needed windows around my pixel in each thread (One thread == One pixel == 4 windows) to get the lowest standard deviation of brightness.

To calculate that standard deviation (SD), we firstly have to get the average value of every window. 

\begin{lstlisting}[style=CStyle]
// Get average V value
for (int x = 1 - windowSize; x <= windowSize - 1; x++){
	for (int y = 1 - windowSize; y <= windowSize - 1; y++){
		
		[...]

		littleWindows[i] += input[loopTid];
		lwPxCount[i]++;
	}
}
for (int i = 0; i < 4; i ++){
    littleWindows[i] /= lwPxCount[i];
}
\end{lstlisting}

After what we simply have to use that value to have SD value.

\begin{lstlisting}[style=CStyle]
// Get SD value

for (int x = 1 - windowSize; x <= windowSize - 1; x++){
	for (int y = 1 - windowSize; y <= windowSize - 1; y++){
		
		[...]

        littleWindowsSd[i] += pow((input[loopTid] - littleWindows[i]), 2.0);
	}
}
for (int i = 0; i < 4; i ++){
      littleWindowsSd[i] = sqrt(littleWindowsSd[i] / lwPxCount[i]);
}
\end{lstlisting}

\subsection{Third kernel : Applying filter to image}

Finally, knowing which window have the lowest SD, we can process each pixel by applying the average value of all pixel on that pixel.

\begin{lstlisting}[style=CStyle]
for (){
    // Pre-processing px
    lwAverageColor[0] += input[loopTid].x;
    lwAverageColor[1] += input[loopTid].y;
    lwAverageColor[2] += input[loopTid].z;	
}

out[tid].x = lwAverageColor[0];
out[tid].y = lwAverageColor[1];
out[tid].z = lwAverageColor[2];
\end{lstlisting}

\section{Explain and measure speedup, if you have performance optimizations}

\subsection{Reducing number of kernel}

The first thing was to reduce the number of kernel.

Even if each kernel are small and it could improve performance, the memory transfer between them was a performance lost. So I merge my kernels into a big one.

Merging kernels wasn't a bad idea, but I have to be very careful with my variables.

\subsection{Algorithm optimization}

Because each thread haven't much memory space, I had to optimize all my variables inside my big kernel. If I do not and one of my memory is too big, it will be moved to a bigger space level ({\it shared}, {\it global}, else) and it will be a performance lost.

\end{document}
