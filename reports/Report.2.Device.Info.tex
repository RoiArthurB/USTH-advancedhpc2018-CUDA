% !TEX TS-program = pdflatex
% !TEX encoding = UTF-8 Unicode

\documentclass[11pt]{article} % use larger type; default would be 10pt

\usepackage[utf8]{inputenc} % set input encoding (not needed with XeLaTeX)

%%% PAGE DIMENSIONS
\usepackage{geometry}
\geometry{a4paper}

\usepackage{graphicx} % support the \includegraphics command and options

% \usepackage[parfill]{parskip} % Activate to begin paragraphs with an empty line rather than an indent

%%% PACKAGES
\usepackage{booktabs} % for much better looking tables
\usepackage{array} 	  % for better arrays (eg matrices) in maths
\usepackage{paralist} % very flexible & customisable lists (eg. enumerate/itemize, etc.)
\usepackage{verbatim} % adds environment for commenting out blocks of text & for better verbatim
\usepackage{subfig}   % make it possible to include more than one captioned figure/table in a single float
\usepackage{listings} % for code presentation
% These packages are all incorporated in the memoir class to one degree or another...

%%% HEADERS & FOOTERS
\usepackage{fancyhdr} % This should be set AFTER setting up the page geometry
\pagestyle{fancy} % options: empty , plain , fancy
\renewcommand{\headrulewidth}{0pt} % customise the layout...
\lhead{}\chead{}\rhead{} % Setup Header
\lfoot{}\cfoot{\thepage}\rfoot{} % Setup Footer

%%% SECTION TITLE APPEARANCE
\usepackage{sectsty}
\allsectionsfont{\sffamily\mdseries\upshape} % (See the fntguide.pdf for font help)
% (This matches ConTeXt defaults)

%%% ToC (table of contents) APPEARANCE
%\usepackage[nottoc,notlof,notlot]{tocbibind} % Put the bibliography in the ToC
%\usepackage[titles,subfigure]{tocloft} % Alter the style of the Table of Contents
%\renewcommand{\cftsecfont}{\rmfamily\mdseries\upshape}
%\renewcommand{\cftsecpagefont}{\rmfamily\mdseries\upshape} % No bold!

%%% END Article customizations

%%% The "real" document content comes below...

\title{Report 2}
\author{Arthur BRUGIERE}
%\date{} % Activate to display a given date or no date (if empty),
         % otherwise the current date is printed 

\begin{document}
\maketitle

\section{Capture output of your labwork regarding GPU information}

\lstset{language=Bash}
\begin{lstlisting}
arthurbrugiere@ictserver3:~/advancedhpc2018/labwork/build$ ./labwork 2
USTH ICT Master 2018, Advanced Programming for HPC.
Warming up...
Starting labwork 2
Number total of GPU : 2

Device Number : 0
Device Name : GeForce GTX 1080
	Clock Rate (KHz): 0.000000
	Core Number : 2560
	Multiprocessor Number : 20
	Warp Size : 32
	Memory Clock Rate (KHz): 0.000000
	Memory Bus Width (bits): 256
	Peak Memory Bandwidth (GB/s): 320.320000

Device Number : 1
Device Name : GeForce GTX TITAN Black
	Clock Rate (KHz): 320.320000
	Core Number : 2880
	Multiprocessor Number : 15
	Warp Size : 32
	Memory Clock Rate (KHz): 320.320000
	Memory Bus Width (bits): 384
	Peak Memory Bandwidth (GB/s): 336.000000

labwork 2 ellapsed 0.9ms
\end{lstlisting}


\end{document}
