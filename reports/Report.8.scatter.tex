% !TEX TS-program = pdflatex
% !TEX encoding = UTF-8 Unicode

\documentclass[11pt]{article} % use larger type; default would be 10pt

\usepackage[utf8]{inputenc} % set input encoding (not needed with XeLaTeX)

%%% PAGE DIMENSIONS
\usepackage{geometry}
\geometry{a4paper}

\usepackage{graphicx} % support the \includegraphics command and options

% \usepackage[parfill]{parskip} % Activate to begin paragraphs with an empty line rather than an indent

%%% PACKAGES
\usepackage{booktabs} % for much better looking tables
\usepackage{array} 	  % for better arrays (eg matrices) in maths
\usepackage{paralist} % very flexible & customisable lists (eg. enumerate/itemize, etc.)
\usepackage{verbatim} % adds environment for commenting out blocks of text & for better verbatim
\usepackage{subfig}   % make it possible to include more than one captioned figure/table in a single float

% These packages are all incorporated in the memoir class to one degree or another...

%%% HEADERS & FOOTERS
\usepackage{fancyhdr} % This should be set AFTER setting up the page geometry
\pagestyle{fancy} % options: empty , plain , fancy
\renewcommand{\headrulewidth}{0pt} % customise the layout...
\lhead{}\chead{}\rhead{} % Setup Header
\lfoot{}\cfoot{\thepage}\rfoot{} % Setup Footer

%%% SECTION TITLE APPEARANCE
\usepackage{sectsty}
\allsectionsfont{\sffamily\mdseries\upshape} % (See the fntguide.pdf for font help)
% (This matches ConTeXt defaults)

%%% ToC (table of contents) APPEARANCE
%\usepackage[nottoc,notlof,notlot]{tocbibind} % Put the bibliography in the ToC
%\usepackage[titles,subfigure]{tocloft} % Alter the style of the Table of Contents
%\renewcommand{\cftsecfont}{\rmfamily\mdseries\upshape}
%\renewcommand{\cftsecpagefont}{\rmfamily\mdseries\upshape} % No bold!

%%% DEV & CODE 
\usepackage{xcolor}
\usepackage{listings} % for code presentation

\definecolor{mGreen}{rgb}{0,0.6,0}
\definecolor{mGray}{rgb}{0.5,0.5,0.5}
\definecolor{mPurple}{rgb}{0.58,0,0.82}
\definecolor{backgroundColour}{rgb}{0.95,0.95,0.92}

\lstdefinestyle{CStyle}{
    backgroundcolor=\color{backgroundColour},   
    commentstyle=\color{mGreen},
    keywordstyle=\color{magenta},
    numberstyle=\tiny\color{mGray},
    stringstyle=\color{mPurple},
    basicstyle=\footnotesize,
    breakatwhitespace=false,         
    breaklines=true,                 
    captionpos=b,                    
    keepspaces=true,                 
    numbers=left,                    
    numbersep=5pt,                  
    showspaces=false,                
    showstringspaces=false,
    showtabs=false,                  
    tabsize=2,
    language=C
}

%%% END Article customizations

%%% The "real" document content comes below...

\title{Report 8}
\author{Arthur BRUGIERE}
%\date{} % Activate to display a given date or no date (if empty),
         % otherwise the current date is printed 

\begin{document}
\maketitle

\section{Explain how you implement the labworks}

The first step was to create an Structure of Array (SoA) to populate it with the HSV parameters of my image

\begin{lstlisting}[style=CStyle]
typedef struct hsv {
    double *h, *s, *v;
} Hsv ;
\end{lstlisting}

After that, I had to allocate memory to that structure on the GPU side. To allocate memory on the good way, I have to allocate memory to each array in my structure (and not only to my structure). 

\begin{lstlisting}[style=CStyle]
Hsv hsvArray;

// Malloc arrays inside the structure
cudaMalloc((void**)&hsvArray.h, pixelCount * sizeof(double));
cudaMalloc((void**)&hsvArray.s, pixelCount * sizeof(double));
cudaMalloc((void**)&hsvArray.v, pixelCount * sizeof(double));

	[...]

cudaFree(hsvArray.h);
cudaFree(hsvArray.s);
cudaFree(hsvArray.v);
\end{lstlisting}

For the other part of this labwork, nothing really important to tell. I implemented what was written in the slides. That labwork is also a Mapping processing algorithm, so {\it One thread = One pixel}.

\section{Explain and measure speedup, if you have performance optimizations}

\subsection{RGB2HSV}

The first thing was to save input pixel to local variables. This way, the thread won't be limited by memory speed limit of the image during the pixel processing. Also, values are pre-processing to gain some nano-seconds.

\begin{lstlisting}[style=CStyle]
double pixelR = (double)in[tid].x / 255.0;
double pixelG = (double)in[tid].y / 255.0;
double pixelB = (double)in[tid].z / 255.0;
\end{lstlisting}

The second thing was to optimize my variable usage (the less I use, the better my algorithm will be). I create a variable only if a same value will be called/processed/other more than once, and I reuse variable if I have no more use of their previous values.\footnote{I know and I agree that it leaves to a less readable code, but my only focus was to optimize my code and my memory usage. So it was a good way to achieve it.}

The third was to process my algorithm only if it needed :

The case is, if a pixel is black, my algorithm will saw it before processing the pixel (because {\it S = 0}) and end with a black pixel.

\begin{lstlisting}[style=CStyle]
if( pxMax <= 0.0 ) { // NOTE: if Max is == 0, this divide would cause a crash
	// if max is 0, then r = g = b = 0              
	// s = 0, h is undefined
	out.h[tid] = 0.0;
	out.s[tid] = 0.0;
	out.v[tid] = 0.0;

	return;
}
\end{lstlisting}

\subsection{HSV2RGB}

The first thing was to save pixel value to local variables.

\begin{lstlisting}[style=CStyle]
double pixelH = in.h[tid];
double pixelS = in.s[tid];
double pixelV = in.v[tid];
\end{lstlisting}

The second thing was to optimize my variable usage (same as before).

\end{document}
