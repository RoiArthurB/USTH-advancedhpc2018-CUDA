% !TEX TS-program = pdflatex
% !TEX encoding = UTF-8 Unicode

\documentclass[11pt]{article} % use larger type; default would be 10pt

\usepackage[utf8]{inputenc} % set input encoding (not needed with XeLaTeX)

%%% PAGE DIMENSIONS
\usepackage{geometry}
\geometry{a4paper}

\usepackage{graphicx} % support the \includegraphics command and options

% \usepackage[parfill]{parskip} % Activate to begin paragraphs with an empty line rather than an indent

%%% PACKAGES
\usepackage{booktabs} % for much better looking tables
\usepackage{array} 	  % for better arrays (eg matrices) in maths
\usepackage{paralist} % very flexible & customisable lists (eg. enumerate/itemize, etc.)
\usepackage{verbatim} % adds environment for commenting out blocks of text & for better verbatim
\usepackage{subfig}   % make it possible to include more than one captioned figure/table in a single float
\usepackage{listings} % for code presentation
% These packages are all incorporated in the memoir class to one degree or another...

%%% HEADERS & FOOTERS
\usepackage{fancyhdr} % This should be set AFTER setting up the page geometry
\pagestyle{fancy} % options: empty , plain , fancy
\renewcommand{\headrulewidth}{0pt} % customise the layout...
\lhead{}\chead{}\rhead{} % Setup Header
\lfoot{}\cfoot{\thepage}\rfoot{} % Setup Footer

%%% SECTION TITLE APPEARANCE
\usepackage{sectsty}
\allsectionsfont{\sffamily\mdseries\upshape} % (See the fntguide.pdf for font help)
% (This matches ConTeXt defaults)

%%% ToC (table of contents) APPEARANCE
%\usepackage[nottoc,notlof,notlot]{tocbibind} % Put the bibliography in the ToC
%\usepackage[titles,subfigure]{tocloft} % Alter the style of the Table of Contents
%\renewcommand{\cftsecfont}{\rmfamily\mdseries\upshape}
%\renewcommand{\cftsecpagefont}{\rmfamily\mdseries\upshape} % No bold!

%%% END Article customizations

%%% The "real" document content comes below...

\title{Report 4}
\author{Arthur BRUGIERE}
%\date{} % Activate to display a given date or no date (if empty),
         % otherwise the current date is printed 

\begin{document}
\maketitle

\section{Explain how you improve the labwork}

Implementing a 2D block changed a few elements.

\subsection{Into the CPU function}

First of all, we have to create the 2D block. We will create one of 32 threads on both dimension (32*32 = 1024 threads, as we used in the previous implementation).

\lstset{language=C}
\begin{lstlisting}
dim3 blockSize = dim3(32, 32);
\end{lstlisting}

After than, we have to calculate the size of our grid. The calcul is a little more complicated because we take care to have enough blocks for processing all the image (even if all threads won't be used in the last block). 

\lstset{language=C}
\begin{lstlisting}
dim3 gridSize = dim3((inputImage->width + (blockSize.x-1))/blockSize.x, 
    (inputImage->height  + (blockSize.y-1))/blockSize.y);
\end{lstlisting}

And, finally, we call our kernel as previously.

\lstset{language=C}
\begin{lstlisting}
grayscale2D<<<gridSize, blockSize>>>(devInput, devGray, inputImage->width, inputImage->height);
\end{lstlisting}

\subsection{Into the GPU function (kernel)}

In a 2D block, we have to calculate the Thread ID. To do it, we have firstly to calculate the id on the x axis and on the y axis. After that, we just have to sum them and we have the global Thread ID to process the image. 

\lstset{language=C}
\begin{lstlisting}
__global__ void grayscale2D(uchar3 *input, uchar3 *output, int imgWidth, int imgHeight) {
    //Calculate tid
    int tidx = threadIdx.x + blockIdx.x * blockDim.x;
    int tidy = threadIdx.y + blockIdx.y * blockDim.y;
    if (tidx >= imgWidth || tidy >= imgHeight) return;

    int tid =  tidx + (tidy * imgWidth);
    [...]
}
\end{lstlisting}

\section{Try experimenting with different 2D block size values}

\section{Plot a graph of block size vs speedup}

\section{Compare speedup with previous 1D grid}

\section{Answer the questions in the upcoming slides, explain why}

\subsection{What is the best configuration for thread blocks to implement grayscaling?}

32x32 is the best implementation. Because, this way, we use the maximum number of threads in each block (1024 threads/block).

\subsection{Which of the following block configs would result in the most number of threads in the SM?}

512 threads/blk = 4*512 = 2,048 threads || 256 threads/blk = 4*256 = 1,024 threads

The best implementation is 256 threads/block because that way we use the maximum of threads in each block without using more than the maximum number of threads that the SM can handle (1,536 threads).

\subsection{Which of the following block configs would result in the most number of threads in the SM?}

2000/512 = 3.91

Because we have to handle all the vector, the grid will use 4 blocks (but 48 threads won't be use).

\end{document}
