% !TEX TS-program = pdflatex
% !TEX encoding = UTF-8 Unicode

\documentclass[11pt]{article} % use larger type; default would be 10pt

\usepackage[utf8]{inputenc} % set input encoding (not needed with XeLaTeX)

%%% PAGE DIMENSIONS
\usepackage{geometry}
\geometry{a4paper}

\usepackage{graphicx} % support the \includegraphics command and options

% \usepackage[parfill]{parskip} % Activate to begin paragraphs with an empty line rather than an indent

%%% PACKAGES
\usepackage{booktabs} % for much better looking tables
\usepackage{array} 	  % for better arrays (eg matrices) in maths
\usepackage{paralist} % very flexible & customisable lists (eg. enumerate/itemize, etc.)
\usepackage{verbatim} % adds environment for commenting out blocks of text & for better verbatim
\usepackage{subfig}   % make it possible to include more than one captioned figure/table in a single float

% These packages are all incorporated in the memoir class to one degree or another...

%%% HEADERS & FOOTERS
\usepackage{fancyhdr} % This should be set AFTER setting up the page geometry
\pagestyle{fancy} % options: empty , plain , fancy
\renewcommand{\headrulewidth}{0pt} % customise the layout...
\lhead{}\chead{}\rhead{} % Setup Header
\lfoot{}\cfoot{\thepage}\rfoot{} % Setup Footer

%%% SECTION TITLE APPEARANCE
\usepackage{sectsty}
\allsectionsfont{\sffamily\mdseries\upshape} % (See the fntguide.pdf for font help)
% (This matches ConTeXt defaults)

%%% ToC (table of contents) APPEARANCE
%\usepackage[nottoc,notlof,notlot]{tocbibind} % Put the bibliography in the ToC
%\usepackage[titles,subfigure]{tocloft} % Alter the style of the Table of Contents
%\renewcommand{\cftsecfont}{\rmfamily\mdseries\upshape}
%\renewcommand{\cftsecpagefont}{\rmfamily\mdseries\upshape} % No bold!

%%% DEV & CODE 
\usepackage{xcolor}
\usepackage{listings} % for code presentation

\definecolor{mGreen}{rgb}{0,0.6,0}
\definecolor{mGray}{rgb}{0.5,0.5,0.5}
\definecolor{mPurple}{rgb}{0.58,0,0.82}
\definecolor{backgroundColour}{rgb}{0.95,0.95,0.92}

\lstdefinestyle{CStyle}{
    backgroundcolor=\color{backgroundColour},   
    commentstyle=\color{mGreen},
    keywordstyle=\color{magenta},
    numberstyle=\tiny\color{mGray},
    stringstyle=\color{mPurple},
    basicstyle=\footnotesize,
    breakatwhitespace=false,         
    breaklines=true,                 
    captionpos=b,                    
    keepspaces=true,                 
    numbers=left,                    
    numbersep=5pt,                  
    showspaces=false,                
    showstringspaces=false,
    showtabs=false,                  
    tabsize=2,
    language=C
}

%%% END Article customizations

%%% The "real" document content comes below...

\title{Report 3}
\author{Arthur BRUGIERE}
%\date{} % Activate to display a given date or no date (if empty),
         % otherwise the current date is printed 

\begin{document}
\maketitle

\section{Explain how you implement the labwork}

The first thing that I have done to implement that labwork is to create and allocate memory for GPU's variable.

\begin{lstlisting}[style=CStyle]
// Declare variable for CUDA Kernel
uchar3 *devInput;
uchar3 *devGray;

// Allocate CUDA memory
cudaMalloc(&devInput, pixelCount * sizeof(uchar3));
cudaMalloc(&devGray, pixelCount * sizeof(uchar3));

// Copy CUDA Memory from CPU to GPU
cudaMemcpy(devInput, inputImage->buffer, pixelCount * sizeof(uchar3), cudaMemcpyHostToDevice);
\end{lstlisting}

After that, I have to create a function (that I will call {\it kernel}) with the prefix {\it \_\_global\_\_}. That function is the one which will be executed by each thread on the GPU.

That function take in input the image buffer, and create in output the image grayscaled. Each thread will process only 1 pixel of the image.

\begin{lstlisting}[style=CStyle]
__global__ void grayscale(uchar3 *input, uchar3 *output) {
    int tid = threadIdx.x + blockIdx.x * blockDim.x;
    output[tid].x = (input[tid].x + input[tid].y + input[tid].z) / 3;
    output[tid].z = output[tid].y = output[tid].x;
}
\end{lstlisting}

Finally, I have to call that kernel in my labwork function by specifying the size of my block, of my grid and the thread's number of each blocks.

\begin{lstlisting}[style=CStyle]
// Start GPU processing (KERNEL)
int blockSize = 1024;
int numBlock = pixelCount / blockSize;
grayscale<<<numBlock, blockSize>>>(devInput, devGray);
\end{lstlisting}

Finally I have to copy back my image output from the GPU memory to the CPU memory.

\begin{lstlisting}[style=CStyle]
// Allocate CPU memory for output image
outputImage = static_cast<char *>(malloc(pixelCount * 3));

// Copy CUDA Memory from GPU to CPU
cudaMemcpy(outputImage, devGray, pixelCount * sizeof(uchar3), cudaMemcpyDeviceToHost);
\end{lstlisting}

Before finishing my function, I have to free all the memory use by the CPU and GPU (here, only for the GPU).

\begin{lstlisting}[style=CStyle]
// Free CUDA Memory
cudaFree(&devInput);
cudaFree(&devGray);
\end{lstlisting}

%\section{What’s the speedup?}
%
%\section{Try experimenting with different block size values}
%
%\section{Plot a graph of block size vs speedup}
%
%\section{Discuss the graph}

\end{document}
